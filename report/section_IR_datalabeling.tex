\subsection{Data labeling and cleansing}
\textbf{Student Name: }Youcef Mammar \textbf{Student ID:} number\\

On Google+ users can label their posts explicitly through \#hashtags to make it easy to retrieve 
all the posts related to an event, topic or tv program. And when they don’t use hashtags, 
scanning the keywords inside each post can return precious information about its subject.

\subsubsection*{Motivation}
The motivation is to use the hashtags and some keywords to label the posts into normalized categories 
in order to allow searching and classification based on a subject.

\subsubsection*{Problem formulation}

\subsubsection*{Approach}
We create categories and give each a set of related keywords. A keyword can be either a regular word of 
the post, or a hashtagged word. Then we use a quite straight forward boolean retrieval technique to 
feed a dictionary of categories which for each category, indexes the related posts. \\
The categories and their related keywords are store in a csv file of this architecture:

\begin{verbatim}
 iphone|technology
 android|technology
 bmw|automotive
 ...
\end{verbatim}

The detection relies on the spell checking as a preprocessing step. As we deal with social content, it is 
best to lowercase everything, ignoring acronyms and proper nouns. We think that users are usually careless 
about conserving the case of such words on social media.

\subsubsection{Evalutation}


